\section{Limits and Continuity}
\label{sect:limits-and-cont}
\begin{enumerate}
\item In MATH2241, we have learnt the concept of \emph{limits and continuity}
in \(\R\). Here, we would extend the notion to a more general \emph{metric
space}. As we will see, some definitions and terminologies are analogous to the
ones we learnt in MATH2241.
\item Throughout \cref{sect:limits-and-cont}, we shall consider a metric space
\((X,d)\).
\end{enumerate}
\subsection{Convergence in a Metric Space}
\begin{enumerate}
\item A sequence \(\{x_n\}\) in \(X\) \defn{converges} to \(a\in X\) if for any
\(\varepsilon>0\), there exists \(N\in\N\) such that
\[
d(x_n,a)<\varepsilon\quad\text{for any \(n\ge N\), \(n\in\N\).}
\]
In this case, we write \(\displaystyle \lim_{n\to \infty}x_n=a\) or
\(\{x_n\}\to a\). We call \(a\) as \defn{limit} of \(\{x_n\}\). If the sequence
\(\{x_n\}\) converges to \emph{some} \(a\in X\), we say that the sequence is
\defn{convergent}. Otherwise, it is called \defn{divergent}.

\begin{note}
The definition of a \defn{sequence} is analogous to the one in MATH2241: It is
a function from \(\N\) to \(X\).
\end{note}

\item \label{it:ms-conv-relate-real-conv} We can relate the notion of
convergence in metric space with that in \(\R\) as follows:
\[\{x_n\}\to a\iff \{d(x_n,a)\}\to 0.\]
\begin{pf}
Note that
\begin{align*}
&\hspace{1cm}\{x_n\}\to a\\
&\iff \text{for any \(\varepsilon>0\), there exists \(N\in\N\) such that \(|d(x_n,a)-0|=d(x_n,a)<\varepsilon\) for any \(n\ge N\)}\\
&\iff \{d(x_n,a)\}\to 0.
\end{align*}
\end{pf}
\item Like the case for \(\R\), the limit here is unique also (if exists). This
property can be proved using a similar approach as the one in MATH2241
(essentially just replacing some symbols).
\begin{proposition}
\label{prp:limit-uniqueness}
A sequence \(\{x_n\}\) in \(X\) can converge to at most one point in \(X\).
\end{proposition}
\begin{pf}
Assume that \(\{x_n\}\to a\) and \(\{x_n\}\to b\) for some \(a,b\in X\).  Fix
any \(\varepsilon>0\). Then there exist \(N_1,N_2\in \N\) such that
\[
d(x_n,a)<\frac{\varepsilon}{2}\quad\text{for any \(n\ge N_1\)},
\]
and
\[
d(x_n,a)<\frac{\varepsilon}{2}\quad\text{for any \(n\ge N_2\)}.
\]
Then, choose \(N=\max\{N_1,N_2\}\). By triangle inequality (M3), for any \(n\ge
N\),
\[
d(a,b)\le d(x_n,a)+d(x_n,b)<\frac{\varepsilon}{2}+\frac{\varepsilon}{2}=\varepsilon.
\]
\end{pf}
\item The following result gives some properties of a convergent sequence.
\begin{proposition}
\label{prp:conv-seq-prop}
Suppose that \(\{x_n\}\to a\) in \(X\). Then,
\begin{enumerate}
\item \(\{x_n:n\in\N\}\) is bounded in \(X\).
\item \(a\) is an adherent point of \(\{x_n:n\in\N\}\).
\item If \(\{x_n:n\in\N\}\) is an infinite set, then \(a\) is an accumulation
point of \(\{x_n:n\in\N\}\).
\end{enumerate}
\end{proposition}
\begin{pf}
\begin{enumerate}
\item 
\end{enumerate}
\end{pf}
\end{enumerate}
