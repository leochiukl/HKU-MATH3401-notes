\section{Limits and Continuity}
\label{sect:limits-and-cont}
\begin{enumerate}
\item In MATH2241, we have learnt the concept of \emph{limits and continuity}
in \(\R\). Here, we would extend the notion to a more general \emph{metric
space}. As we will see, some definitions and terminologies are analogous to the
ones we learnt in MATH2241.
\end{enumerate}
\subsection{Convergence in a Metric Space}
\label{subsect:conv-ms}
\begin{enumerate}
\item Throughout \Cref{subsect:conv-ms}, we shall consider an arbitrary metric
space \((X,d)\).
\item A sequence \(\{x_n\}\) in \(X\) \defn{converges} to \(a\in X\) if for any
\(\varepsilon>0\), there exists \(N\in\N\) such that for any positive integer
\(n\ge N\),
\[
d(x_n,a)<\varepsilon,
\]
i.e., \(x_n\in B(a,\varepsilon)\).  In this case, we write \(\displaystyle
\lim_{n\to \infty}x_n=a\), \(\{x_n\}\to a\), or \(\{x_n\}_{n=1}^{\infty}\to
a\). We call \(a\) as \defn{limit} of \(\{x_n\}\). If the sequence \(\{x_n\}\)
converges to \emph{some} \(a\in X\), we say that the sequence is
\defn{convergent}. Otherwise, it is called \defn{divergent}.

\begin{note}
The definition of a \defn{sequence} is analogous to the one in MATH2241: It is
a function from \(\N\) to \(X\).
\end{note}

\item \label{it:ms-conv-relate-real-conv} We can relate the notion of
convergence in metric space with that in \(\R\) as follows:
\[\{x_n\}\to a\iff \{d(x_n,a)\}\to 0.\]
\begin{pf}
Note that
\begin{align*}
&\hspace{1cm}\{x_n\}\to a\\
&\iff \text{for any \(\varepsilon>0\), there exists \(N\in\N\) such that \(|d(x_n,a)-0|=d(x_n,a)<\varepsilon\) for any \(n\ge N\)}\\
&\iff \{d(x_n,a)\}\to 0.
\end{align*}
\end{pf}
\item Like the case for \(\R\), the limit here is unique also (if exists). This
property can be proved using the same approach as the one in MATH2241
(essentially just replacing some symbols).
\begin{proposition}
\label{prp:limit-uniqueness}
A sequence \(\{x_n\}\) in \(X\) can converge to at most one point in \(X\).
\end{proposition}
\begin{pf}
Assume that \(\{x_n\}\to a\) and \(\{x_n\}\to b\) for some \(a,b\in X\).  Fix
any \(\varepsilon>0\). Then there exist \(N_1,N_2\in \N\) such that
\[
d(x_n,a)<\frac{\varepsilon}{2}\quad\text{for any \(n\ge N_1\)},
\]
and
\[
d(x_n,a)<\frac{\varepsilon}{2}\quad\text{for any \(n\ge N_2\)}.
\]
Then, choose \(N=\max\{N_1,N_2\}\). By triangle inequality (M3), for any \(n\ge
N\),
\[
d(a,b)\le d(x_n,a)+d(x_n,b)<\frac{\varepsilon}{2}+\frac{\varepsilon}{2}=\varepsilon.
\]
\end{pf}
\item The following result gives some properties of a convergent sequence.
\begin{proposition}
\label{prp:conv-seq-prop}
Suppose that \(\{x_n\}\to a\) in \(X\). Then,
\begin{enumerate}
\item \(\{x_n:n\in\N\}\) is bounded in \(X\).
\item \(a\) is an adherent point of \(\{x_n:n\in\N\}\).
\item If \(\{x_n:n\in\N\}\) is an infinite set, then \(a\) is an accumulation
point of \(\{x_n:n\in\N\}\).
\end{enumerate}
\end{proposition}
\begin{pf}
\begin{enumerate}
\item When \(\{x_n\}\to a\), there exists \(N\in\N\) such that \(d(x_n,a)<1\)
for any \(n\ge N\). Then, by setting
\(M=\max\{d(x_1,a),\dotsc,d(x_{N-1},a),1\}\), we have \(d(x_n,a)<M\) for any
\(n\in\N\). Applying triangle inequality (M3), we can then show that
\(\{x_n:n\in\N\}\) is bounded in \(X\).

\item Fix any \(r>0\). By assumption there exists \(N\in\N\) such that \(x_n\in
B(a,r)\) for any \(n\ge N\). Thus, \(B(a,r)\cap\{x_n:n\in\N\}\ne\varnothing\),
so \(a\) is an adherent point of \(\{x_n:n\in\N\}\).

\item Fix any \(r>0\). By assumption there exists \(N\in\N\) such that \(x_n\in
B(a,r)\) for any \(n\ge N\). Since \(\{x_n:n\in\N\}\) is infinite, the
intersection \(B(a,r)\cap\{x_n:n\in\N\}\supseteq \{x_n:n\ge N\}\) is also
infinite. Hence, by \Cref{prp:acc-pt-inf-set}, \(a\) is an accumulation point
of \(\{x_n:n\in\N\}\).
\end{enumerate}
\end{pf}
\item Using the concept of \emph{sequences}, we can obtain an useful
characterization of adherent points and accumulation points as follows:
\begin{proposition}
\label{prp:adher-accum-seq-equiv}
Let \(S\subseteq X\) and \(a\in X\). Then,
\begin{enumerate}
\item \(a\in\overline{S}\) iff there is a sequence \(\{x_n\}\) in \(S\) such
that \(\{x_n\}\to a\).
\item \(a\in S'\) iff there is an infinite sequence \(\{x_n\}\) of distinct
points in \(S\) such that \(\{x_n\}\to a\).
\end{enumerate}
\end{proposition}
\begin{pf}
\begin{enumerate}
\item ``\(\Rightarrow\)'': Suppose that \(a\in \overline{S}\). Then for any
\(n\in\N\), there exists \(x_n\in S\) such that \(\displaystyle
0\le d(x_n,a)<\frac{1}{n}\). We can see that the sequence \(\{x_n\}\) constructed in
this way converges to \(a\), since \(\{d(x_n,a)\}\to 0\) by sandwich theorem.

``\(\Leftarrow\)'': Assume that there is a sequence \(\{x_n\}\) in \(S\)
converging to \(a\). By \Cref{prp:conv-seq-prop}, we know that \(a\) is an
adherent point of \(\{x_n:n\in\N\}\), i.e., \(a\in\overline{\{x_n:n\in\N\}}\).
Since the sequence \(\{x_n\}\) is in \(S\), we have \(\{x_n:n\in\N\}\subseteq
S\). It then follows by \labelcref{it:clos-pre-subset} that
\[a\in\overline{\{x_n:n\in\N\}}\subseteq \overline{S}.\]

\item ``\(\Rightarrow\)'': Suppose that \(a\in S'\). Note first that for any
\(n\in\N\), by \Cref{prp:acc-pt-inf-set}, \(B(a,1/n)\cap S\) is an infinite
set. Thus, it is possible to choose \(x_n\in S\setminus
\{x_1,\dotsc,x_{n-1}\}\) with \(\displaystyle 0\le d(x_n,a)<\frac{1}{n}\) for
every \(n\in\N\). This constructs an infinite sequence \(\{x_n\}\) of distinct
points in \(S\) with \(\{x_n\}\to a\).

``\(\Leftarrow\)'': Assume there is an \emph{infinite} sequence \(\{x_n\}\) of
distinct points in \(S\) such that \(\{x_n\}\to a\). By
\Cref{prp:conv-seq-prop}, we know that \(a\in\{x_n:n\in\N\}'\). Since
\(\{x_n:n\in\N\}\subseteq S\), it follows by \labelcref{it:deriv-pre-subset}
that
\[
a\in\{x_n:n\in\N\}'\subseteq S'.
\]
\end{enumerate}
\end{pf}
\item With this sequential characterization, we can get yet another criterion
for closedness.

\begin{proposition}
\label{prp:closed-seq-crit}
Let \(S\subseteq X\). Then, the following are equivalent.
\begin{enumerate}
\item \(S\) is closed in \(X\).
\item If \(\{x_n\}\) is a sequence in \(S\) with \(\{x_n\}\to a\), then \(a\in
S\).
\end{enumerate}
\end{proposition}
\begin{pf}
\underline{\(\text{(a)}\implies \text{(b)}\)}: Suppose that \(S\) is closed in \(X\) and
\(\{x_n\}\) is a sequence in \(S\) with \(\{x_n\}\to a\). Then by
\Cref{prp:adher-accum-seq-equiv} and \Cref{prp:equiv-crit-closed}, we have
\(a\in \overline{S}=S\).

\underline{\(\text{(b)}\implies \text{(a)}\)}: Assume that (b) holds. Fix any
\(a\in\overline{S}\). By \Cref{prp:adher-accum-seq-equiv}, there is a sequence
\(\{x_n\}\) in \(S\) such that \(\{x_n\}\to a\). Applying (b) then gives \(a\in
S\). This shows \(\overline{S}\subseteq S\). Since we must have \(S\subseteq
\overline{S}\), it follows that \(\overline{S}=S\), hence \(S\) is closed in
\(X\) by \Cref{prp:equiv-crit-closed}.
\end{pf}
\item Next, we will discuss a result about \emph{subsequences}, which is
a generalization to the respective result in MATH2241. It can be proved using
the same approach as the one in MATH2241.
\begin{proposition}
\label{prp:subseq-same-lim}
Let \(\{x_n\}\) be a sequence in \(X\). Then \(\{x_n\}\to a\) iff
\(\{x_{n_k}\}_{k=1}^{\infty}\to a\) for every subsequence \(\{x_{n_k}\}\) of
\(\{x_n\}\).
\end{proposition}
\begin{note}
The concept of \emph{subsequence} is defined in the same manner as the one for
MATH2241: We first let \(n_1<n_2<n_3<\dotsb\) be positive integers sorted in
strictly increasing order, and then the sequence \(\{x_{n_k}\}_{k=1}^{\infty}\)
is a \defn{subsequence} of \(\{x_n\}\).
\end{note}

\begin{pf}
``\(\Leftarrow\)'' is immediate since \(\{x_n\}\) is a subsequence of itself.
So it remains to prove ``\(\Rightarrow\)''. Assume that \(\{x_n\}\to a\). For
any \(\varepsilon>0\), there exists \(N\in\N\) such that
\(d(x_n,a)<\varepsilon\) for any \(n\ge N\). Fix any subsequence
\(\{x_{n_k}\}\) of \(\{x_n\}\). Then for any \(k\ge N\), since \(n_k\ge k\ge
N\), we have \(d(x_{n_k},a)<\varepsilon\).
\end{pf}
\end{enumerate}
\subsection{Complete Metric Spaces}
\label{subsect:complete-ms}
\begin{enumerate}
\item Throughout \Cref{subsect:complete-ms}, we shall consider an arbitrary
metric space \((X,d)\).

\item To motivate the notion of \emph{complete metric spaces}. We consider the
set \(\Q\) of rational numbers. It is well-known that
\(\sqrt{2}\in\R\setminus\Q\) is \emph{irrational}. Nonetheless, using decimal
representation, we can obtain better and better \emph{approximations} to
\(\sqrt{2}\), using just rational numbers.

More specifically, we can express \(\sqrt{2}\) as
\(\fpeval{sqrt(2)}\cdots\). From this we can construct a sequence \(\{x_n\}\)
in \(\Q\) as follows:
\[
x_1=\fpeval{round(sqrt(2),0)},\quad
x_2=\fpeval{round(sqrt(2),1)},\quad
x_3=\fpeval{round(sqrt(2),2)},\quad
x_4=\fpeval{round(sqrt(2),3)},\quad
x_5=\fpeval{round(sqrt(2),4)},\quad
x_6=\fpeval{round(sqrt(2),5)},\quad
x_7=\fpeval{round(sqrt(2),6)},\quad\cdots
\]
by rounding \(\sqrt{2}\) in increasing number of decimal places. We can show
that \(\{x_n\}\to\sqrt{2}\notin\Q\). So, this sequence \(\{x_n\}\) does
\emph{not} converge to any number in \(\Q\).

However, it is not hard to observe that the terms in this sequence get ``closer
and closer'' together, so intuitively it seems that the sequence should
converge to a certain ``point''. Here that ``point'' is \(\sqrt{2}\), which is
\emph{not} contained in \(\Q\). So in this sense, \(\Q\) appears to have
something ``missing'' at \(\sqrt{2}\) and it is not so ``complete''.

\item This kind of sequence with terms getting ``closer and closer'' together
can be characterized by \emph{Cauchy sequence}. \begin{note} We have studied
this concept in the special case of \(\R\) in MATH2241. Here we consider a more
general one. \end{note}

A sequence \(\{x_n\}\) in \(X\) is said to be a \defn{Cauchy sequence} if for
any \(\varepsilon>0\), there exists \(N\in\N\) such that
\(d(x_n,x_m)<\varepsilon\) for any \(n,m\ge N\).

\item As a \emph{convergent} sequence also has the feature that the terms get
``closer and closer'' together, intuitively it seems that it should also be a
Cauchy sequence. This is indeed the case.

\begin{proposition}
\label{prp:conv-imp-cauchy}
Every convergent sequence is Cauchy.
\end{proposition}
\begin{pf}
Consider any convergent sequence \(\{x_n\}\to a\) in \(X\). Then for any
\(\varepsilon>0\), there exists \(N\in\N\) such that
\[
d(x_n,a)<\frac{\varepsilon}{2}\quad\text{for any \(n\ge N\)}.
\]
Thus, for any \(n,m\ge N\), by triangle inequality (M3),
\[
d(x_n,x_m)\le d(x_n,a)+d(x_m,a)<\varepsilon\quad\text{for any \(n,m\ge N\)}.
\]
Thus, \(\{x_n\}\) is Cauchy.
\end{pf}

\item Intuitively, a Cauchy sequence ``stabilizes'' as the index gets larger.
Hence, naturally we would expect that the \emph{distances} between terms
respectively taken from two Cauchy sequences ``stabilize'' also. This intuitive
idea is proven below.

\begin{proposition}
\label{prp:dist-cauchy-conv}
Let \(\{x_n\}\) and \(\{y_n\}\) be two Cauchy sequences. Then, the sequence
\(\{d(x_n,y_n)\}\) converges in \(\R\).
\end{proposition}
\begin{pf}
Since the sequence \(\{d(x_n,y_n)\}\) is in \(\R\), it suffices to prove that
the sequence \(\{d(x_n,y_n)\}\) is Cauchy.

Fix any \(\varepsilon>0\). Since \(\{x_n\}\) and \(\{y_n\}\) are Cauchy, there
exist \(N_1,N_2\in\N\) such that
\begin{align*}
d(x_n,x_m)&<\frac{\varepsilon}{2}\quad\text{for any \(n,m\ge N_1\)}\\
d(y_n,y_m)&<\frac{\varepsilon}{2}\quad\text{for any \(n,m\ge N_2\)}.
\end{align*}
We then choose \(N=\max\{N_1,N_2\}\). Then, for any \(n,m\ge N\), we have
\begin{align*}
|d(x_n,y_n)-d(x_m,y_m)|
&= |d(x_n,y_n)\vc{-d(x_n,y_m)+d(x_n,y_m)}-d(x_m,y_m)|\\
&\le |d(x_n,y_n)\vc{-d(x_n,y_m)}|+|\vc{d(x_n,y_m)}-d(x_m,y_m)|&\text{(triangle inequality)}\\
&\le |d(y_n,y_m)|+|d(x_n,x_m)|&\text{(M3)}\\
&\le d(y_n,y_m)+d(x_n,x_m)\\
&<\frac{\varepsilon}{2}+\frac{\varepsilon}{2}\\
&=\varepsilon.
\end{align*}
This shows that \(\{d(x_n,y_n)\}\) is Cauchy, as desired.
\end{pf}

\item In our previous \(\sqrt{2}\) example, the rational sequence there can be
shown to be Cauchy but not convergent. So a Cauchy sequence is not necessarily
convergent. Indeed, the convergence of Cauchy sequence allows us to
characterize the notion of ``completeness''. Intuitively, divergence of a
Cauchy sequence indicates a ``missing point''. To be ``complete'', there should
be no ``missing point'', thus every Cauchy sequence should be convergent.

A metric space \((X,d)\) is said to be \defn{complete} if \emph{every} Cauchy
sequence in \(X\) converges in \(X\). A subset \(S\) of \(X\) is said to be
\defn{complete} if, when considered as a metric space itself, \((S,d)\) is
complete.

\item Next, we shall establish an important result that the Euclidean space
\(\R^k\) is complete for any \(k\in\N\).

\begin{theorem}
\label{thm:rk-complete}
The metric space \(\R^k\) is complete for any \(k\in\N\).
\end{theorem}
\begin{pf}
We shall base our proof on the fact that \(\R\) is complete. Consider any
Cauchy sequence \(\{(x_n^{(1)},\dotsc,x_n^{(k)})\}\) in \(\R^k\). Note that for
every \(i=1,\dotsc,k\), \(\{x_n^{(i)}\}\) is a Cauchy sequence in \(\R\). Then
by the completeness of \(\R\), we must have \(\{x_n^{(i)}\}\to a^{(i)}\) for
some \(a^{(i)}\in\R\). This then implies that
\[
\{(x_n^{(1)},\dotsc,x_n^{(k)})\}\to (a^{(1)},\dotsc,a^{(k)})\in\R^k.
\]
\end{pf}

\item The next result connects \emph{compactness} with \emph{completeness}. It
turns out that compactness, being a rather strong condition, does implies
completeness.

\begin{theorem}
\label{thm:cpt-subset-comp}
Every compact subset \(S\) of a metric space \(X\) is complete.
\end{theorem}
\begin{pf}
Let \(\{x_n\}\) be any Cauchy sequence in \(S\).

\underline{Case 1}: \(\{x_n:n\in\N\}\) is a finite set.

Then there exists \(N\in\N\) such that \(x_n=x\in S\) for any \(n\ge N\). This
shows \(\lim_{n\to \infty}x_n=x\in S\).
\begin{note}
For this case, we do even need compactness!
\end{note}

\underline{Case 2}: \(\{x_n:n\in\N\}\) is an infinite set.

By \Cref{thm:cpt-bw-prop}, \(X\) has the Boltzano-Weierstrass property. Hence,
as an infinite subset of \(X\), the set \(\{x_n:n\in\N\}\) has an accumulation
point \(a\) in \(\{x_n:n\in\N\}\). Fix any \(\varepsilon>0\). First, since
\(\{x_n\}\) is Cauchy, there exists \(N\in\N\) such that
\(d(x_n,x_m)<\varepsilon/2\) for any \(n,m\ge N\). Also, since \(a\) is an
accumulation point of \(\{x_n:n\in\N\}\), the intersection
\(B(a,\varepsilon/2)\cap\{x_n:n\in\N\}\) is infinite. Thus, there exists \(m\ge
N\) such that \(x_m\in B(a,\varepsilon/2)\).

Now, applying triangle inequality, we have
\[
d(x_n,a)\le d(x_n,x_m)+d(x_m,a)<\frac{\varepsilon}{2}+\frac{\varepsilon}{2}=\varepsilon
\]
for any \(n\ge N\). This shows \(\{x_n\}\to a\), as desired.
\end{pf}

\subsubsection*{Construction of Completion (Optional)}
\item Given a metric space \((X,d)\) which may not be complete, we can
construct the \emph{completion} of \((X,d)\), denoted by
\((\widetilde{X},\widetilde{d})\). The \defn{completion} of \((X,d)\) is the
smallest complete metric space containing \(X\), i.e., for any complete metric
space \((Y,d_Y)\) with \(Y\supseteq X\), the set \(\widetilde{X}\) is a subset
of \(Y\) (with possibly some ``embedding''). \begin{note}
The metric \(d_Y\) is the same as \(d\) except that its domain is changed to
\(Y\times Y\).
\end{note}

Example: The completion of \(\Q\) equipped with standard Euclidean metric is
\(\R\) with the same metric.

\item Here we will demonstrate how to actually construct such completion.
As suggested previously, every Cauchy sequence in \(X\) may be seen as
corresponding to a ``point'', to which the terms are ``getting closer''. The
intuitive idea is then to include all the Cauchy sequences in \(X\) in the
completion, so that all the possible corresponding ``points'' are included ---
There is not ``missing point''!

\item However, it is not hard to notice that doing in this way would result in
many redundancies. It appears that many Cauchy sequences actually correspond to
the same ``point''. For example, both the Cauchy sequences \(\{0.9, 0.99,
0.999, 0.9999, \dotsc\}\) and \(\{1.1, 1.01, 1.001, 1.0001, \dotsc\}\) appear
to correspond to the same ``point'': \(1\). But we want the completion to be
``smallest'', so all those redundancies should be removed.

\item This leads us to consider the idea of \emph{equivalence classes}, to
``represent'' each ``point'' by using only one element, namely the
\emph{equivalence class} containing all the ``equivalent'' Cauchy sequences
corresponding to the same ``point''.

Hence, we would like to define an equivalence relation \(\sim\) on the set of
all Cauchy sequences in \(X\) to capture our intuitive idea of ``corresponding
to the same point''.

\item To define such equivalence relation, we first introduce the following
function. For any two Cauchy sequences \(\{x_n\}\) and \(\{y_n\}\) in \(X\),
define the function
\(\widetilde{\delta}:X\times X\to\R\) by
\[\widetilde{\delta}(\{x_n\},\{y_n\})=\lim_{n\to \infty}d(x_n,y_n).\]
Note that the limit \(\lim_{n\to \infty}d(x_n,y_n)\) always exists by
\Cref{prp:dist-cauchy-conv}, so the function \(\widetilde{\delta}\) is well-defined.

\item Next, we define \(\{x_n\}\) and \(\{y_n\}\) as \emph{equivalent}, written
as \(\{x_n\}\sim\{y_n\}\), if \(\widetilde{\delta}(\{x_n\},\{y_n\})=0\).

\begin{intuition} When the distance between two Cauchy sequences are zero, the
terms in each Cauchy sequence should ``get closer'' to the same ``point''.
\end{intuition}


\item We now let
\[
\widetilde{X}=\{\text{all equivalence classes of Cauchy sequences in \(X\)}\}.
\]
Note that \(\widetilde{\delta}(\{x_n\},\{y_n\})\) is invariant after replacing
either of the sequences by a Cauchy sequence equivalent to it.

\begin{pf}
WLOG, we only prove that
\(\widetilde{\delta}(\{x_n\},\{y_n\})=\widetilde{\delta}(\{x_n\},\{z_n\})\), where
\(\{z_n\}\) be any Cauchy sequence equivalent to \(\{y_n\}\), i.e.,
\(\{z_n\}\sim\{y_n\}\).

Using (M3) of the underlying metric \(d\), we have
\[
\widetilde{\delta}(\{x_n\},\{y_n\})\le
\widetilde{\delta}(\{x_n\},\{z_n\})+\underbrace{\widetilde{\delta}(\{z_n\},\{y_n\})}_{0}
=\widetilde{\delta}(\{x_n\},\{z_n\}).
\]
On the other hand, using (M3) of the metric \(d\) again, we get
\[
\widetilde{\delta}(\{x_n\},\{z_n\})\le
\widetilde{\delta}(\{x_n\},\{y_n\})+\underbrace{\widetilde{\delta}(\{y_n\},\{z_n\})}_{0}
=\widetilde{\delta}(\{x_n\},\{y_n\}).
\]
This shows that
\(\widetilde{\delta}(\{x_n\},\{y_n\})=\widetilde{\delta}(\{x_n\},\{z_n\})\).
\end{pf}

With this property, the following function \(\widetilde{d}:\widetilde{X}\times
\widetilde{X}\to\R\) is well-defined:
\[
\widetilde{d}([s_1],[s_2])=\widetilde{\delta}(s_1,s_2)
\]
where \(s_1=\{x_n\}\) and \(s_2=\{y_n\}\) are any elements taken from the equivalence classes
\([s_1]\) and \([s_2]\) in \(\widetilde{X}\) respectively.


\item We can then verify that \(\widetilde{d}\) is indeed a metric as follows.

\begin{pf}
(M1): Due to the nonnegativity of the underlying metric \(d\), we must have
\(d([s_1],[s_2])=\widetilde{\delta}(s_1,s_2)\ge 0\). Next, observe that
\(\widetilde{\delta}(s_1,s_2)=0\) iff \(s_1=s_2\), due to the underlying metric
\(d\). Since equivalence classes are either equal or disjoint, it follows that
\(d([s_1],[s_2])=0\) iff \([s_1]=[s_2]\).

(M2): It follows from (M2) of the underlying metric \(d\):
\(d(x_n,y_n)=d(y_n,x_n)\) for any \(n\in\N\).

(M3): It follows from (M3) of the underlying metric \(d\).
\end{pf}

\item Now we define an \emph{embedding} \(\iota:X\to\widetilde{X}\) by
\(\iota(x)=[\{x,x,x,\dotsc\}]\), i.e., the equivalence class containing the
sequence \(\{x_n\}\) with \(x_n=x\) for any \(n\in\N\).

The embedding \(\iota\) is an \emph{isometry} or \emph{distance-preserving
map}, i.e., \(d(x,y)=\widetilde{d}(\iota(x),\iota(y))\) for any \(x,y\in X\).
Then, by identifying every \(x\in X\) with the corresponding embedding
\(\iota(x)\in\widetilde{X}\) (``treating them as the same''), we may say that
\(X\) is a ``subset'' of \(\widetilde{X}\) (with embedding).

\item Next, we will show that \((\widetilde{X},\widetilde{d})\) is indeed a
\emph{complete} metric space. We want to show that every Cauchy sequence of
equivalence classes in \(\widetilde{X}\) converges to a certain equivalence class in
\(\widetilde{X}\).

Fix any Cauchy sequence \(\{[s_n]\}_{n=1}^{\infty}\) of equivalence classes in
\(\widetilde{X}\), where \(s_n\) is a Cauchy sequence
\(\{x_m^{(n)}\}_{m=1}^{\infty}\) in \(X\) for any \(n\in\N\).

To prove the convergence of \(\{[s_n]\}_{n=1}^{\infty}\), we will use a special
kind of diagonal argument.  First fix any \(k\in\N\). Since
\(s_k=\{x_m^{(k)}\}_{m=1}^{\infty}\) is Cauchy in \(X\), there exists
\(N_k\in\N\) such that
\[
d\qty(x_{m_1}^{(k)},x_{m_2}^{(k)})<\frac{1}{k}
\]
for any \(m_1,m_2\ge N_k\). We denote the \(N_k\)th term in the sequence
\(s_k\), namely \(x_{N_k}^{(k)}\), by \(y_k\). Then particularly we would have
\begin{equation}
\label{eq:completion-dxky}
d\qty(x_{m_1}^{(k)},y_k)=d\qty(x_{m_1}^{(k)},x_{N_k}^{(k)})<\frac{1}{k}
\end{equation}
for any \(m_1\ge N_k\).

We claim that the sequence \(\{[s_n]\}\) converges to the equivalence class
containing the sequence
\(\{y_k\}_{k=1}^{\infty}=\{x_{N_k}^{(k)}\}_{k=1}^{\infty}\), which is in \(\widetilde{X}\).

\begin{pf}
Firstly, we shall prove that \(\{y_k\}_{k=1}^{\infty}\) is a Cauchy sequence in
\(X\), which shows that the equivalence class containing
\(\{y_k\}_{k=1}^{\infty}\) is indeed in \(\widetilde{X}\).  Fix any
\(\varepsilon>0\). Then, we choose a sufficiently large \(N\in\N\) such that:
\begin{enumerate}
\item \(1/N<\varepsilon/3\), and
\item Considering any \(m,n\ge N\), there should exist a sufficiently large
\(j\in\N\) such that \(d(x_j^{(n)},x_j^{(m)})<\varepsilon/3\) since
\(\{[s_i]\}_{i=1}^{\infty}\) is Cauchy, which implies that
\(\widetilde{\delta}(\{x_j^{(n)}\},\{x_j^{(m)}\})=\lim_{j\to
\infty}d(x_j^{(n)},x_j^{(m)})\) can be arbitrarily small, as long as \(m,n\)
are large enough.
\end{enumerate}
Then, for any \(m,n\ge N\) with a sufficiently large \(j\in\N\), that exceeds
both \(N_m\) and \(N_n\), and makes \(d(x_j^{(n)},x_j^{(m)})<\varepsilon/3\),
we have
\[
d(y_n,y_m)=d(x_{N_n}^{(n)},x_{N_m}^{(m)})
\le
d(x_{N_n}^{(n)},x_j^{(n)})+d(x_j^{(n)},x_j^{(m)})+d(x_j^{(m)},x_{N_m}^{(m)})
<\frac{1}{n}+\frac{\varepsilon}{3}+\frac{1}{m}
\le\frac{\varepsilon}{3}+\frac{\varepsilon}{3}+\frac{\varepsilon}{3}
=\varepsilon,
\]
establishing the Cauchyness of \(\{y_k\}_{k=1}^{\infty}\) in \(X\).

Next, we want to show that \(\{[s_n]\}\to [\{y_k\}_{k=1}^{\infty}]\). In other
words, we need to show that for any \(\varepsilon>0\), there exists \(N\in\N\)
such that
\[
\widetilde{d}([s_n],[\{y_k\}_{k=1}^{\infty}])
=\widetilde{\delta}(s_n,\{y_k\}_{k=1}^{\infty})
=\widetilde{\delta}(\{x_k^{(n)}\}_{k=1}^{\infty},\{y_k\}_{k=1}^{\infty})
=\lim_{k\to \infty}d(x_k^{(n)},y_k)
=\lim_{k\to \infty}d(x_k^{(n)},x_{N_k}^{(k)})
<\varepsilon
\]
for any \(n\ge N\).

We first fix any \(\varepsilon>0\), and then choose a sufficiently large
\(N\in\N\) such that (i) \(1/N<\varepsilon/2\) and (ii) for any \(m,n\ge N\),
we have \(d(y_n,y_m)=d(x_{N_n}^{(n)},x_{N_m}^{(m)})<\varepsilon/2\). Then, for any
\(n\ge N\), we have

\begin{align*}
\widetilde{d}([s_n],[\{y_k\}_{k=1}^{\infty}])
&=\lim_{k\to \infty}d(x_k^{(n)},x_{N_k}^{(k)})\\
&\le\lim_{k\to \infty}d(x_k^{(n)},x_{N_n}^{(n)})+\lim_{k\to \infty}d(x_{N_n}^{(n)},x_{N_k}^{(k)}) \\
&<\frac{1}{n}+\frac{\varepsilon}{2}&\text{(using \labelcref{eq:completion-dxky} for the first term; (ii) for the second term)} \\
&\le\frac{1}{N}+\frac{\varepsilon}{2} \\
&<\frac{\varepsilon}{2}+\frac{\varepsilon}{2} \\
&=\varepsilon,
\end{align*}
completing the proof.
\end{pf}
\item Finally, we shall show that \((\widetilde{X},\widetilde{d})\) is the
\emph{smallest} complete metric space containing \(X\). Consider another
complete metric space \((Y,d_Y)\) containing \(X\) also, where
\(d_Y\) is the same as \(d\) except that the domain
becomes \(Y\times Y\).

Note that for every class \([s]\in\widetilde{X}\), the Cauchy sequence \(s\) in
\(X\) is also a Cauchy sequence in \(Y\), thus converges to some \(y\in Y\) by
the completeness of \(Y\). From this, define a function \(\rho:\widetilde{X}\to
Y\) by \(\rho([s])=y\).

We claim that the function \(\rho\) is an isometry, and so the set
\(\widetilde{X}\) can be regarded as a subset of \(Y\) with ``embedding''.

\begin{pf}
Consider any two classes \([s_1],[s_2]\in\widetilde{X}\). Suppose the
corresponding Cauchy sequences \(s_1\) and \(s_2\) converge to \(y_1\) and
\(y_2\) (both in \(Y\)) respectively.

Since the definitions of \(d_Y\) and \(d\) coincide on \(X\times  X\), we may
extend the definition of \(\widetilde{\delta}\) through replacing the
underlying metric \(d\) by \(d_Y\), and changing its domain to \(Y\times Y\).
Correspondingly, the domain of metric \(\widetilde{d}\) can be extended to
\(\widetilde{Y}\times\widetilde{Y}\) where \(\widetilde{Y}\) is the set of all
equivalence classes of Cauchy sequences in \(Y\).

Now, to show that \[\widetilde{d}([s_1],[s_2])=d_Y(\rho([s_1]),\rho([s_2]))
=d_Y(y_1,y_2),\] we first pick the Cauchy sequences \(\{y_1,y_1,\dotsc\}\in
[s_1]\) and \(\{y_2,y_2,\dotsc\}\in [s_2]\).\footnote{We have
\(\widetilde{\delta}(\{y_1,y_1,\dotsc\},s_1)=0\) since \(s_1\to y_1\); similar
for another sequence.} Then, we have
\[
\widetilde{d}([s_1],[s_2])=\widetilde{\delta}(\{y_1,y_1,\dotsc\},\{y_2,y_2,\dotsc\})
=d_Y(y_1,y_2),
\]
as desired.
\end{pf}
\end{enumerate}
\subsection{Continuous Functions}
\label{subsect:cts-fun}
\begin{enumerate}
\item As we have mentioned at the very beginning, in MATH3401 we are studying
continuous functions between metric spaces. We have analyzed metric spaces in
\Cref{sect:metric-spaces}. It is now time to study \emph{continuous functions},
another central concept in MATH3401.

\item The concept of continuous functions is utilized for studying the
``equivalence problem'' in metric space topology, namely determining whether
two metric spaces are ``equivalent'' in some sense. We will introduce a notion
called \emph{homeomorphism}, which is related to the ``equivalence'' of metric
spaces. Simply speaking, homeomorphism is a bijective continuous function whose
inverse is also continuous.

Throughout \Cref{subsect:cts-fun}, we shall use the notations \((X,d_{X})\) and
\((Y,d_Y)\) to denote arbitrary metric spaces. Let us first generalize the
\(\varepsilon\)-\(\delta\) definition of \emph{limit of function} studied in
MATH2241 to the context of metric spaces.

\item Let \(S\) be a subset of \(X\), and \(f:S\to Y\) be a function. Suppose
that \(p\in X\) is an accumulation point of \(S\). Then, we write
\(\displaystyle \lim_{x\to p}f(x)=b\) or \(f(x)\to b\) as \(x\to p\) if for any
\(\varepsilon>0\), there exists \(\delta>0\) such that
\[
d_Y(f(x),b)<\varepsilon
\]
for any \(x\in S\setminus\{p\}\) with \(d_X(x,p)<\delta\), or in other words,
\[
f(x)\in B_Y(b,\varepsilon)
\]
for any \(x\in B_X(p,\delta)\cap S\setminus \{p\}\), or more compactly:
\[
f\big((B_X(p,\delta)\cap S\setminus \{p\})\big)\subseteq B_Y(b,\varepsilon).
\]
\item Like MATH2241, we have the following sequential criterion for limits of
functions.
\begin{proposition}
\label{prp:lim-seq-crit}
Let \(p\in X\) be an accumulation point of \(S\subseteq X\), \(b\in Y\), and
\(f:S\to Y\) be any function. Then, \(\displaystyle \lim_{x\to p}f(x)=b\) iff
\(\displaystyle \lim_{n\to \infty}f(x_n)=b\) for every sequence \(\{x_n\}\) in
\(S\setminus\{p\}\) which converges to \(p\).
\end{proposition}
\begin{pf}
TODO
\end{pf}

\item Next, we will consider a result applicable when the codomain is \(\R^n\),
related to the Euclidean norm \(\|\cdot\|\).
\begin{proposition}
\label{prp:lim-norm}
Let \(S\) be a subset of \(X\), \(p\in X\) be an accumulation point of \(S\),
and \(f:S\to\R^n\) be any function. Then,
\[
\lim_{x\to p}f(x)=b\implies \lim_{x\to p}\|f(x)\|=\|b\|,
\]
where \(\|y\|=d_E(y,\vect{0})\) for any \(y\in\R^n\) (\(\vect{0}\) is
the zero vector in \(\R^n\) and \(d_E\) is the Euclidean metric).
\end{proposition}
\begin{pf}
TODO
\end{pf}
\end{enumerate}
